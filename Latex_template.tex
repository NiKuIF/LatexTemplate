\documentclass[a4paper, 11pt]{article}

% % % % % % % % % % % % % % Usepackages % % % % % % % % % % % % % % % % % % % % % %
\usepackage[utf 8]{inputenc}		% fuer Umlaute
\usepackage{amsmath}						% Mathebibliothek, notwendig für Matrizenschreibweise
\usepackage{tikz}								% fuer Tikz-Skizzen
\usepackage{graphicx}						% um Bilder einzufügen
\usepackage{fancyhdr}						% fue Fancy-Header notwendig
\usepackage[T1]{fontenc}				% for encoding of special charackters with out escaping them
\usepackage[left=3.1cm,
	right=2.5cm,
	top=3.6cm,
	bottom=2cm,
	footskip=25pt,
	includeheadfoot]{geometry}		% Seitenrändern für geometry package anpassen

\usepackage{caption}[2008/08/24]			% fuer Bilderbeschriftung
\usepackage{pifont}							% fue checkmarks and other symbols
\usepackage{amssymb}
\usepackage{german}
\usepackage[pages=some]{background}	% to allow background images

% Header einfuegen, ueber fancy-Header
\pagestyle{fancy} 				% eigener Seitenstil
\fancyhf{} 						% alle Kopf- und Fußzeilenfelder bereinigen
\lhead{Nistelberger}		% Linker Header Teil
\chead{Softwareparadigmen}	% Mittlerer Header Teil
\rhead{Hausübung 1}					% Rechter Header Teil
\rfoot{Seite \thepage}
\renewcommand{\headrulewidth}{0.4pt} %obere Trennlinie setzen
\renewcommand{\footrulewidth}{0.4pt} %untere Trennlinie

% % % % % % % % % % % % % % Macros % % % % % % % % % % % % % % % % % % % % % % % % %
% Gibt die uebergebene Zahl als roemische Zahl zurueck
\newcommand{\RM}[1]{\MakeUppercase{\romannumeral #1{.}}}
\newcommand{\tab}[1]{\hspace{.1\textwidth}\rlap{#1}}

% macros for checkmarks
\newcommand{\cmark}{\ding{51}}
\newcommand{\cmarkL}{{\Large \ding{51}}}
\newcommand{\xmark}{\ding{55}}

% nützliche Befehle:
% \noindent
% \hspace*{10mm}
% \glqq text \grqq
% $\rightarrow$
% $\square$

%\begin{figure}[!htbp]
%	\begin{center}
%		\includegraphics[scale=0.22]{image}
%		\caption{ARP Cache}
%		\label{fig:reference_label}
%	\end{center}
%\end{figure}

\backgroundsetup{
	scale=1,
	color=black,
	opacity=1,
	angle=0,
	contents={%
		\includegraphics[width=\paperwidth,height=\paperheight]{titlepage.pdf}
	}%
}


% create titlepage
\begin{titlepage}
	\centering
	%\includegraphics[width=0.15\textwidth]{example-image-1x1}\par\vspace{1cm}
	{\scshape\LARGE TU Graz \par}
	\vspace{1cm}
	{\scshape\Large Hausübung.1\par}
	\vspace{1.5cm}
	{\huge\bfseries Softwareparadigmen\par}
	\vspace{2cm}
	{\Large\itshape Stefan Papst 1430868\par}
	{\Large\itshape Kurt Nistelberger 1431233\par}
	\vfill
	supervised by\par
	Försten \textsc{Ruprechter}

	\vfill

% Datum in der Fußzeile
	{\large \today\par}
\end{titlepage}

% create Titlepage with Image and write name on/over the image
% geometry and restoregeometry caused a lot of troubles so i switched to background images
%\begin{titlepage}
%\BgThispage	% start page with background image here
%\begin{picture}(0,300)        % choose size of picture for the content
%	%\put(x,y){content}
%	\put(115,204){{\Large Nistelberger}}
%	\put(115,172){{\Large 1431233}}
%\end{picture}
%\clearpage		% end page with background image
%\end{titlepage}

% Inhaltsverzeichnis, bei Änderungen muss zweimal compiliert werden
\tableofcontents
\newpage

% Start of the real Document

\section{Beispiel}
\subsection{a}

\end{document}
