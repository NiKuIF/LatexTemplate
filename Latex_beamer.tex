\documentclass{beamer}

% ********************************
% Presentation for Internship 2016
% ********************************

\usepackage{subfigure}


\title{Synthesizer on AIG Graphs}
\author{Internship 2016, Kurt Nistelberger}

\begin{document}

% creates the title page
\maketitle
 
 
\begin{frame}

{\LARGE Goal of the Internship:} \\

\begin{itemize}
\item learn how a Synthesizer work
\item What are Safety Games?
\item What is AIGER?
\item What is FRAIG?
\item Deal with ABC
\end{itemize}

\end{frame}


\begin{frame}
\frametitle{Brief Introduction about Synthesizer}


%einbinden einer Grafik
\begin{figure}
    \subfigure[Intput specification file to the synthesizer]{\includegraphics[scale=0.32]{spec.png} }
    \subfigure[Output produced by synthesizer]{\includegraphics[scale=0.32]{output.png} }
%\caption{Titel unterm gesamten Bild}
\end{figure}

Transition Function:\\
\begin{center}
$T = trans( x, c, i, x' ) \rightarrow T/F$  \\
\end{center}

\textbf{Aim:}\\
{\small
Find a circuit which controls the controllable inputs such that the
error output will always stay FALSE, no matter what the sequence of
uncontrollable inputs is.}

\end{frame}


\begin{frame}
\frametitle{Safety Games}

Explain Safety Games shortly plus the algorithm

\end{frame}


\begin{frame}
\frametitle{AIGER}

AIG = AND-Inverter-Graph\\
AIGER = AIG + Latches\\

\end{frame}

\end{document}