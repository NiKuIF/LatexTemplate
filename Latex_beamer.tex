\documentclass{beamer}

% ********************************
% Presentation for Internship 2016
% ********************************

\usepackage[utf 8]{inputenc}
\usepackage{subfigure}
\usepackage{color}


\title{Synthesizer on AIG Graphs}
\author{Internship 2016, Kurt Nistelberger}

\begin{document}

% creates the title page
\maketitle
 
 
\begin{frame}

{\LARGE Goal of the Internship:} \\

\begin{itemize}
\item learn how a Synthesizer work
\item What are Safety Games?
\item What is AIGER?
\item What is FRAIG?
\item Implement a Synthesizer in ABC
\end{itemize}

\end{frame}


\begin{frame}
\frametitle{Brief Introduction about Synthesizer}


\begin{figure}
    \subfigure[Intput specification file to the synthesizer]{\includegraphics[scale=0.32]{spec.png} }
    \subfigure[Output produced by synthesizer]{\includegraphics[scale=0.32]{output.png} }
%\caption{Titel unterm gesamten Bild}
\end{figure}

%Transition Function:\\
%\begin{center}
%$T = trans( x, c, i, x' ) \rightarrow T/F$  \\
%\end{center}

\textbf{Aim:}\\
{\small
Find a circuit which controls the controllable inputs such that the
error output will always stay FALSE, no matter what the sequence of
uncontrollable inputs is.}

\end{frame}


%\begin{frame}
%\frametitle{Games}
%Different types of games:\\
%\begin{itemize}
%\item Reachability Games
%\item Safety Games
%\item Büchi Games
%\end{itemize}
%We are going to solve safety games which means, something
%bad never happens.
%\end{frame}


\begin{frame}
\frametitle{Safety Games}


\noindent\begin{minipage}{0.3\textwidth}% adapt widths of minipages to your needs
\includegraphics[scale=0.3]{../../Documents/Internship_TU/saftey_games_states.png}
\end{minipage}%
\hfill%
\begin{minipage}{0.4\textwidth}\raggedleft
Exclude all states so that 
we never leave the \\winning region.
\end{minipage}

\bigskip
\bigskip

\textbf{squares:} controlled by the environment, want to reach a bad state\\ 
\textbf{circles:} we can control, we want to stay in the winning region \\

\end{frame}

\begin{frame}
\frametitle{Winning Region Algorithm}

\textbf{def} $Cpre(dst_{(s)}):$ \\
\noindent\hspace*{6mm} $dst'_{(s)} = dst_{(s)}|$ s is replaced by s'\\
\noindent\hspace*{6mm} \textbf{return} $\forall i \exists c \exists s': trans( s, c, i, s' ) \wedge dst(s')$\\

\bigskip

\textbf{def} $calc\_win\_region(): $\\
\noindent\hspace*{6mm} $cur_{(s)} = True$ \\
\noindent\hspace*{6mm} \textbf{while} $True:$ \\
\noindent\hspace*{12mm} $new_{(s)} = Cpre(cur_{(s)})$ \& $cur_{(s)} $ \\
\noindent\hspace*{12mm} \textbf{if} $ new_{(s)} == cur_{(s)}$ \\
\noindent\hspace*{18mm} \textbf{return} $cur_{(s)}$\\
\noindent\hspace*{12mm} \textbf{if} $ new_{(s)} == 0$ \\
\noindent\hspace*{18mm} \textbf{return} $0$\\
\noindent\hspace*{12mm} $cur_{(s)} = new_{(s)}$

\end{frame}

\begin{frame}
\frametitle{How to compute a winning region using AIGs}


\noindent\begin{minipage}{0.5\textwidth}% adapt widths of minipages to your needs
We need the following Operations:
\begin{itemize}
\item AND-Operator
\item $\exists$-Quantification
\item $\forall$-Quantification
\item compare two Circuits
\end{itemize}

\end{minipage}%
\hfill%
\begin{minipage}{0.4\textwidth}
\includegraphics[scale=0.4]{../../Documents/Internship_TU/AIG_sample.png} 
\end{minipage}

\end{frame}


\begin{frame}
\frametitle{Operators}

AND-Operator:

\includegraphics[scale=0.4]{../../Documents/Internship_TU/Final_Presentation/AIG_AND.png} 

%\bigskip

$\exists$-Quantification:\\
$ \exists c$ $Q(a, b, c) = Q(a, b, c=\top) \vee Q(a, b, c=\bot) $\\

\bigskip

$\forall$-Quantification:\\
$ \forall c$ $Q(a, b, c) = Q(a, b, c=\top) \wedge Q(a, b, c=\bot) $

\bigskip

compare two Circuits: \\
\begin{itemize}
\item XOR the circuits
\item Fraig the combined circuit
\item check if the result is constant true
\end{itemize}


\end{frame}

% AIGER-FileFormat Slide
%\begin{frame}
%\frametitle{AIG and AIGER}
%\includegraphics[scale=0.3]{../../Documents/Internship_TU/aiger_format.png} 
%\end{frame}

\begin{frame}
\frametitle{FRAIG - Functionally Reduced AIG}

Introduce a new AND-Node \textbf{without} functional reduction,\\
the following Pseudocode is used:\\

\bigskip

\textbf{if} $res=checkTrivialCases(n1, n2) $\\
\noindent\hspace*{6mm} \textbf{return} $res$ \\
\textbf{else if} $res=checkOneLevelStrashing(n1, n2) $\\
\noindent\hspace*{6mm} \textbf{return} $res$ \\
\textbf{else} \\
\noindent\hspace*{6mm} \textbf{return} $createNewNode(n1,n2)$ \\


\end{frame}


\begin{frame}
\frametitle{FRAIG - Functionally Reduced AIG}

Introduce a new AND-Gate \textbf{with} functional reduction,\\
the following steps are used:\\

\bigskip

\textbf{if} $res=checkTrivialCases(n1, n2) $\\
\noindent\hspace*{6mm} \textbf{return} $res$ \\
\textbf{else if} $res=checkOneLevelStrashing(n1, n2) $\\
\noindent\hspace*{6mm} \textbf{return} $res$ \\
\textbf{else} // functional Reduction \\
\noindent\hspace*{6mm} $class = HashtableLookUp(n1,n2)$\\
\noindent\hspace*{6mm} \textbf{if} $class == NULL$\\
\noindent\hspace*{12mm} \textbf{return} $createNewNode(n1,n2)$\\
\noindent\hspace*{6mm} \textbf{else} \\
\noindent\hspace*{12mm} $res = checkEachCandidateInClass(class)$\\
\noindent\hspace*{12mm} \textbf{if} $res == NULL$ \\
\noindent\hspace*{18mm} \textbf{return} $createNewNode(n1,n2)$ \\
\noindent\hspace*{12mm} \textbf{else}  \\
\noindent\hspace*{18mm} \textbf{return} $res$ \\

\end{frame}


\begin{frame}
\frametitle{ABC: A System for Sequential Synthesis and Verification}

ABC is a tool for sequential Synthesis and Verification and provides 
lots of Algorithms for: Model checking, equivalence checking, logic synthesis, 
reachability computation, verification...\\

\bigskip

\textbf{Why we used ABC as base:} \\
Because it also provides many functions to deal with AIG-Files.\\
e.g. Read/Write-Operations, AND-Operator, Variable-Quantification, 
Transition Relation Computation....

%\textbf{Quote:} The goal of the ABC project is to provide a public-domain %implementation of the state-of-the-art combinational and sequential synthesis %algorithms.\\

%\bigskip

%It provides Algorithms for: Model checking, equivalence checking, logic synthesis, 
%reachability computation, verification...\\

%\bigskip

%Advantages why we use ABC as basis:\\

%\begin{itemize}
%\item<1-> provides internal handling of AIGs Read/Write
%\item<2-> provides many useful functions to deal with AIGs
%\item<3-> provides useful functions to compute transition relation and variable quantification
%\end{itemize}


\end{frame}

%\begin{frame}
%\frametitle{ABC: A System for Sequential Synthesis and Verification}

%\bigskip

%Disadvantages of using ABC:\\
%\begin{itemize}
%\item<1-> ABC is a very big project
%\item<2-> provides to many functions for dealing with AIGs
%\item<3-> use three different AIG-Manager internally
%\item<4-> some functions might not behave as you wish (e.g. AND-Operator)
%\end{itemize}

%\end{frame}


\begin{frame}
\frametitle{First Implementation, Results}

\includegraphics[scale=0.19]{../../Documents/Internship_TU/ex1.png} 

\end{frame}


\begin{frame}
\frametitle{First Implementation, Results}

\includegraphics[scale=0.19]{../../Documents/Internship_TU/ex2.png} 

\end{frame}


\begin{frame}
\frametitle{First Implementation, Results}

\includegraphics[scale=0.19]{../../Documents/Internship_TU/ex3.png} 

\end{frame}


\begin{frame}
\frametitle{First Implementation, Results}

\includegraphics[scale=0.19]{../../Documents/Internship_TU/ex4.png} 

\end{frame}


\begin{frame}
\frametitle{First Implementation, Results}

\includegraphics[scale=0.19]{../../Documents/Internship_TU/ex5.png} 

\end{frame}

\begin{frame}
\frametitle{First Implementation, Results}

\includegraphics[scale=0.19]{../../Documents/Internship_TU/ex6.png} 

\end{frame}


\begin{frame}
\frametitle{First Implementation, Results}

\includegraphics[scale=0.19]{../../Documents/Internship_TU/ex7.png} 

\end{frame}

\begin{frame}
\frametitle{Hat-Problem}

The grows after every Quantification Operation and we have to get rid
of hat.\\
I applied several optimizations which were already implemented in ABC.
(and also more..)

\bigskip

\textbf{What have I tried:} \\
\begin{itemize}[label={}]
\item<2-> fraig
\item<3-> rewrite
\item<4-> dc2 ( rewrite, refactor, balance )
\item<5-> dc2 several times
\end{itemize}

\bigskip

\end{frame}

\begin{frame}
\frametitle{Mitigation for the Hat-Problem}

Reduce the number of $\exists$-Quantification calls, by a reduce of
the number of Inputs.\\

\bigskip

Implement Vector Compose.

\bigskip

\textbf{Advantages of Vector Compose:}\\

\begin{itemize}

\item represent every prime variable through its prime AIG
\item reduces the number of primary inputs
\item reduces the number of $\exists$-Quantifications
\item transition relation is no more needed

\end{itemize}

\end{frame}

\begin{frame}
\frametitle{Winning Region Algorithm with VectorCompose}

\textbf{def} $Cpre(dst_{(s)}):$ \\
\noindent\hspace*{6mm} // {\small  \emph{ \color{gray} $dst'_{(s)} = dst_{(s)}|$ s is replaced by s'}}\\
\noindent\hspace*{6mm} // {\small  \emph{ \color{gray}\textbf{return} $\forall i \exists c \exists s': trans( s, c, i, s' ) \wedge dst(s')$}} \\
%\noindent\hspace*{6mm} \textbf{return} $\forall i \exists c \exists s': trans( s, c, i, s' ) \wedge dst(s')$\\
\noindent\hspace*{6mm} \textbf{return} $\forall i \exists c : dst[s’ \leftrightarrow func(s, c, i) ]$\\

\bigskip

\textbf{def} $calc\_win\_region(): $\\
\noindent\hspace*{6mm} $cur_{(s)} = True$  \\
\noindent\hspace*{6mm} \textbf{while} $True:$ \\
\noindent\hspace*{12mm} $new_{(s)} = Cpre(cur_{(s)})$ \& $cur_{(s)} $ \\
\noindent\hspace*{12mm} \textbf{if} $ new_{(s)} == cur_{(s)}$ \\
\noindent\hspace*{18mm} \textbf{return} $cur_{(s)}$\\
\noindent\hspace*{12mm} \textbf{if} $ new_{(s)} == 0$ \\
\noindent\hspace*{18mm} \textbf{return} $0$\\
\noindent\hspace*{12mm} $cur_{(s)} = new_{(s)}$

\end{frame}

\begin{frame}
\frametitle{Motivation for VectorCompose}

\begin{figure}

\includegraphics[scale=0.35]{../../Documents/Internship_TU/time_comp.png} 
\caption{time usage graph}

\end{figure}
\end{frame}

\begin{frame}
\frametitle{Motivation for VectorCompose}

\begin{figure}
\includegraphics[scale=0.5]{../../Documents/Internship_TU/speed_comp.png} 
\caption{speed graph}
\end{figure}
 
\end{frame}


\begin{frame}
\frametitle{Implementation, Results}

\includegraphics[scale=0.35]{../../Documents/Internship_TU/abcVSaisy.png} 

\end{frame}

\begin{frame}
\frametitle{Summary of the Internship}

\textbf{encountered problems with ABC:}\\

\begin{itemize}
\item<1-> Code in ABC is sometimes very hard to read because it is perfomance optimized\\ 
	  (thats why I assumed everything inside ABC works properly)
\item<2-> some functions may not behave as you expect \\
	  (e.g. AND-Operator)
\item<3-> if ABC throwed an assertion checked if the assertion is really necessary
\item<4-> before you implement a function from scratch try to find a similar 			implementation in ABC (cut vs. node)

\end{itemize}

\end{frame}


%\begin{frame}
%\frametitle{Summary of the Internship}
%\begin{itemize}
%\item<1-> the start was hard
%\item<2-> the end was  harder
%\item<3-> it took me a while to deal with ABC 
%\item<4-> the problem with the hat still exists
%\end{itemize}
%\end{frame}


\begin{frame}
\frametitle{Future work}

\begin{itemize}

\item carefully analyse why the circuit grows so fast...\\
\item more effient implementation of Vector Compose

\end{itemize}
\end{frame}



\begin{frame}

\begin{center}
{\LARGE \textbf{Questions?}}
\end{center}

\end{frame}

\end{document}